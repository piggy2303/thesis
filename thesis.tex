\documentclass[12pt]{report}
\usepackage[utf8]{vietnam}
\usepackage{titlesec}
\usepackage{titletoc}
\usepackage{listings}
\usepackage[bookmarks=true]{hyperref}
\usepackage[left=3cm,right=2cm,top=2.5cm,bottom=3cm]{geometry}
\usepackage{graphicx}
\usepackage{hyperref}
\setlength{\parindent}{10mm}
\renewcommand{\baselinestretch}{1.3}

% hyper setup
\hypersetup{
	bookmarks=true,
	pdftitle={Xây dựng công cụ hỗ trợ quản lý và đảm bảo chất lượng cho các phiên bản phần mềm},
	pdfauthor={Bùi Quang Cường}, % author
	pdfsubject={TeX and LaTeX},
	pdfkeywords={TeX, LaTeX, graphics, images}, % list of keywords
	colorlinks=true,       % false: boxed links; true: colored links
	linkcolor=blue,       % color of internal links
	citecolor=black,       % color of links to bibliography
	filecolor=black,        % color of file links
	urlcolor=purple,        % color of external links
	linktoc=page            % only page is linked
}

\date{}


\newpagestyle{long}
{\sethead{\thesection. \sectiontitle}{}{\subsectiontitle}\headrule
	\setfoot{}{\thepage}{}}


\begin{document}
\begin{titlepage}
	\center
	{\large \bfseries ĐẠI HỌC QUỐC GIA HÀ NỘI\\ TRƯỜNG ĐẠI HỌC CÔNG NGHỆ}\\[1cm]
	\includegraphics[width=0.2\linewidth]{images/uet}\\[1cm]
	{\Large  \bfseries Bùi Quang Cường}\\[2cm]
	{ \LARGE \bfseries XÂY DỰNG CÔNG CỤ HỖ TRỢ QUẢN LÝ VÀ ĐẢM BẢO CHẤT LƯỢNG CHO CÁC PHIÊN BẢN PHẦN MỀM}\\[0.5cm]
	\hfill\\[3cm]
	{\large \bfseries KHÓA LUẬN TỐT NGHIỆP ĐẠI HỌC HỆ CHÍNH QUY}\\	
	{\large \bfseries Ngành: Công nghệ thông tin}	
	\hfill\\[3cm]	
	{\large \bfseries HÀ NỘI - 2018}\\	
	\vfill
\end{titlepage}
	
%-----SECONDARY TITLE PAGE-----%	
\begin{titlepage}
	\center
	{\large \bfseries ĐẠI HỌC QUỐC GIA HÀ NỘI\\ TRƯỜNG ĐẠI HỌC CÔNG NGHỆ}\\[1cm]
	\includegraphics[width=0.2\linewidth]{images/uet}\\[1cm]
	{\Large  \bfseries Bùi Quang Cường}\\[2cm]		
	{ \LARGE \bfseries XÂY DỰNG CÔNG CỤ HỖ TRỢ QUẢN LÝ VÀ ĐẢM BẢO CHẤT LƯỢNG CHO CÁC PHIÊN BẢN PHẦN MỀM}\\[0.5cm]
	\hfill\\[3cm]
	{\large \bfseries KHÓA LUẬN TỐT NGHIỆP ĐẠI HỌC HỆ CHÍNH QUY}\\	
	{\large \bfseries Ngành: Công nghệ thông tin}		
	\hfill\\[3cm]		
	{\large \bfseries HÀ NỘI - 2018}\\		
	\vfill		
\end{titlepage}

%-----THANKS-----%
\newpage
\begin{titlepage}
\begin{center}
	\textbf{\large LỜI CẢM ƠN}
\end{center}
\end{titlepage}

	
%-----ABSTRACT-----%
\newpage
\begin{titlepage}
\begin{center}
	\textbf{\large TÓM TẮT}
\end{center}
\textbf{Tóm tắt:} Mã nguồn ứng dụng trở nên lớn và phức tạp sau quá trình dài phát triển, bảo trì và nâng cấp. Vấn đề này đang dần phổ biến đối với các doanh nghiệp phát triển phần mềm, gây khó khăn trong việc kiểm soát và đảm bảo chất lượng cho các ứng dụng. Hiện nay đã có một số công cụ được đề xuất để giải quyết vấn đề trên nhưng chưa có kết quả thỏa đáng. Nghiên cứu này đề xuất các phương pháp và xây dựng một bộ công cụ toàn diện cho việc phân tích và đảm bảo chất lượng mã nguồn cho các ứng dụng doanh nghiệp sử dụng các nền tảng J2EE phổ biến Struts 2, Hibernate, Spring. Đầu tiên, mã nguồn của ứng dụng sẽ được tiền xử lý để tạo cây cấu trúc. Mỗi nút trên cây đại diện cho một thành phần mã nguồn. Tiếp theo, các nút sẽ được phân tích theo các công nghệ sử dụng trong ứng dụng để xác định mối quan hệ phụ thuộc. Cây cấu trúc này được sử dụng làm đầu vào cho việc phân tích ảnh hưởng sự thay đổi và các chức năng phân tích cấu trúc như xây dựng đồ thị dữ liệu; xây dựng kiến trúc về công nghệ, cơ sở dữ liệu; tính toán độ phức tạp mã nguồn. Phương pháp phân tích ảnh hưởng sự thay đổi được đề xuất cải tiến theo hướng tự động hóa bằng phương pháp so sánh các phiên bản mã nguồn. Hiện nay, bộ công cụ đang được triển khai thử nghiệm tại Trung tâm công nghệ và quản lý chất lượng phần mềm Viettel (VITM) và nhận được nhiều phản hồi tích cực.\\

\noindent \textit{\textbf{Từ khóa:} phân tích mã nguồn, phiên bản mã nguồn, ứng dụng doanh nghiệp}
\end{titlepage}

%-----ABSTRACT (ENGLISH)-----%
\newpage
\begin{titlepage}
\begin{center}
	\textbf{\large ABSTRACT}
\end{center}
\end{titlepage}

%-----UNDERTAKING-----%
\newpage
\begin{titlepage}
\begin{center}
	\textbf{\large LỜI CAM ĐOAN}
\end{center}
\end{titlepage}

%-----TOC-----%
\newpage
\begin{titlepage}
\tableofcontents
\end{titlepage}

%-----MAIN-----%
\newpage
\setcounter{page}{1}
\chapter{Đặt vấn đề}
Hiện nay, các ứng dụng tại các doanh nghiệp thường được phát triển trong một thời gian
dài với quy mô lớn và độ phức tạp cao. Trải qua nhiều phiên bản nâng cấp, các ứng dụng
này thường thiếu tài liệu đặc tả và thiết kế. Có thể nói tài liệu gần như duy nhất của các
ứng dụng này là mã nguồn. Trong khi đó, quá trình bảo trì và nâng cấp diễn ra thường
xuyên. Để đảm bảo chất lượng cho mỗi phiên bản mới, đội dự án cần thực hiện kiểm thử
lại toàn bộ hệ thống. Điều này là không thể bởi chi phí cho việc này rất tốn kém. Kết quả
là chúng ta không thể kiếm soát toàn bộ sự ảnh hưởng của việc thay đổi và có thể dẫn đến
nhiều rủi ro lớn cho doanh nghiệp trong quá trình vận hành. Đây là một bài toán khó tổng
quát hóa vì các giải pháp đề xuất phụ thuộc chặt chẽ vào các công nghệ được sử dụng
trong ứng dụng. Đề xuất các giải pháp và xây dựng công cụ đủ tốt để giải quyết vấn đề
nêu trên đang là một trong những thách thức lớn và nhận được sự quan tâm nghiên cứu.\\

Phân tích ảnh hưởng sự thay đổi (Change Impact Analysis - CIA) được xem là một
giải pháp để giải quyết bài toán trên. CIA có vai trò quan trọng trong các giai đoạn phát
triển, bảo trì và kiểm thử hồi quy. Đối với người quản lý và lập trình viên, CIA là công cụ
đánh giá phạm vi ảnh hưởng, ước lượng chi phí từ đó lên kế hoạch thực hiện thay đổi.
Đối với kiểm thử viên, trong quá trình kiểm thử hồi quy, CIA có thể ứng dụng vào việc
xác định những ca kiểm thử có liên quan đến phần mã nguồn chỉnh sửa, điều này giúp
giảm số lượng các ca kiểm thử cần thực hiện từ đó rút ngắn thời gian và nỗ lực kiểm thử.
Các kĩ thuật CIA được thực hiện theo hai hướng tiếp cận chính là phân tích tĩnh (static
CIA) và phân tích động (dynamic CIA) [1]. Trong thực tế, hầu hết các nghiên cứu đề xuất
các phương pháp phân tích ảnh hưởng sự thay đổi liên quan đến static CIA. Nổi bật trong
số đó là kĩ thuật CIA dựa trên sự phân loại thay đổi [2], kĩ thuật dựa vào đồ thị Call
Graph [3] và kĩ thuật dựa trên tư tưởng giao thoa sóng nước WAVE-CIA [4]. Tuy nhiên,
Các kĩ thuật CIA sử dụng đầu vào là kết quả của quá trình phân tích phụ thuộc từ ứng
dụng. Quá trình này không thể tổng quát hóa cho toàn bộ các công nghệ và nền tảng hiện
có.\\

J2EE đang là một giải pháp phổ biến được sử dụng để triển khai cho các ứng dụng
Web doanh nghiệp hiện đại. Nó bao gồm nhiều công nghệ cốt lõi như EJB, JSF, CDI,
JAX-WS, JPA, JMS, v.v và có thể tích hợp nhiều framework khác như Spring, Struts,
Hibernate, Vaddin, GWT, Play!, Grails, v.v. Để giải quyết bài toán trên cho các ứng dụng 
2
J2EE, một phương pháp kèm theo công cụ có tên JCIA [5] đã được đề xuất và xây dựng.
Tuy nhiên phương pháp này khá thô sơ và chỉ đề xuất phân tích cho một số công nghệ cốt
lõi của J2EE. Công cụ JCIA còn đơn giản, chưa được hoàn thiện và chưa mang lại tính
hiệu quả cao.\\

Dựa trên ý tưởng về phương pháp này, một phương pháp hoàn chỉnh đã được nghiên
cứu và đề xuất để thực hiện phân tích ảnh hưởng sự thay đổi cho các ứng dụng J2EE đa
nền tảng. Các framework phổ biến hiện nay: Spring, Struts, Hibernate sẽ được tập trung
hỗ trợ đầu tiên. Sau đó chúng tôi sẽ dần hoàn thiện phương pháp cho các tất cả các nền
tảng khác. Ngoài ra, chúng tôi cũng đề xuất các giải pháp khác để đem lại nhiều góc nhìn
khách quan về hệ thông ứng dụng như xây dựng dòng dữ liệu, tái hiện thiết kế kiến trúc
hệ thống và cơ sở dữ liệu, và tính toán độ phức tạp mã nguồn. Một bộ công cụ được phát
triển như là phiên bản tiếp theo của công cụ JCIA để thực hiện các giải pháp đã đề xuất.\\

Phần còn lại của báo cáo được cấu trúc như sau. Phần 2 trình bày các phương pháp
thực hiện về tiền xử lý mã nguồn; phân tích phụ thuộc cho các công nghệ Struts 2, Spring,
Hibernate; quản lý các phiên bản phân tích; xây dựng dòng dữ liệu và cuối cùng là biểu
diễn các góc nhìn kiến trúc của ứng dụng. Tiếp theo, phần 3 trình bày bộ công cụ đảm
bảo chất lượng JCIA-VT, và kết quả đạt được qua thực nghiệm phân tích. Cuối cùng,
phần 4 tóm tắt các kết quả đạt được, kết luận, hạn chế và hướng nghiên cứu phát triển
trong tương lai.


\newpage	
\chapter{Lý thuyết phân tích ảnh hưởng sự thay đổi}
Định nghĩa phân tích ảnh hưởng sự thay đổi được đề xuất lẫn đầu tiên vào năm 1986 bởi Horowiz và cộng sự: "sự kiểm tra tác động để xác định các thành phần hoặc phần tử của tác động đó". Trong hơn 30 năm qua, đã có nhiều kỹ thuật CIA được nghiên cứu và đề xuất. Một số phương pháp được thực hiện dựa trên sự kiểm tra truy vết (traceability-based CIA) trong khi một số khác tập trung xem xét các mối quan hệ phụ thuộc (dependency-based CIA) để xác định các ảnh hưởng. Kỹ thuật CIA dựa trên phân tích các mối quan hệ phụ thuộc sẽ cố gắng phân tích để xác định các quan hệ...

\newpage
\chapter{Phương pháp tiền xử lý mã nguồn}
\textbf{Định nghĩa:} (\textit{Cây cấu trúc}) Cho mã nguồng ứng dụng J2EE, một cây cấu trúc của mã nguồn này được định nghĩa $T = (V, E)$ với $V = \{v_1, v_2,..., v_k\}$ là tập nút đại diện cho các thành phần trong mã nguồn như thư mục; tệp; lớp, phương thức, thuộc tính (Java); thẻ (XML, JSP); v,v. $E = \{(v_i, v_j) | v_i,v_j \in V\}$ là tập các cạnh. Mỗi cạnh $(v_i,v_j)$ đại diện cho quan hệ phụ thuộc giữa $n_i$ và $n_j$ có nghĩa là $n_i$ phụ thuộc vào $n_j$.\\

Các ứng dụng doanh nghiệp J2EE, ngoài mã nguồn Java còn sử dụng nhiều định
dạng mã nguồn như XML, JSP, XHTML, FreeMarker,...Với mỗi một định dạng lại có
cấu trúc và cú pháp khác nhau. Bộ tiền xử lý cần biến đổi những mã nguồn này về định
dạng chung là cây cấu trúc, các nút trên cây cấu trúc cần chứa những thông tin cần thiết
cho việc phân tích phụ thuộc cũng như cây cần được thiết kế để tối ưu cho việc duyệt và
tìm kiếm các nút trên cây dễ dàng.
\chapter{Phương pháp phân tích phụ thuộc cho các ứng dụng J2EE}
\chapter{Thực nghiệm và triển khai}
\chapter{Kết luận}
\end{document}
